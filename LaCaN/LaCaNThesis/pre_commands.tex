% !TEX root = ../LaCaNtesis.tex
% setter commands

\newcommand{\setAuthor}[1]
{
  \newcommand{\Autor}{#1}
}

\newcommand{\setTitle}[1]
{
  \newcommand{\ttitulo}{#1}
  \newcommand{\Titulo}{{\large{\ttitulo}}}
}

\newcommand{\setAdvisor}[1]
{
  \newcommand{\Advisor}{#1}
}

\newcommand{\setDate}[1]
{
  \newcommand{\Date}{#1}
}

\newcommand{\figuresPath}{}

\newcommand{\setFiguresPath}[1]
{
  \renewcommand{\figuresPath}{#1}
}

% Do's commands

\newcommand{\doTitle}
{
  \pagenumbering{roman}
\pagestyle{plain}

\thispagestyle{empty}

\vspace{-3cm} \centerline{\large \sc{Universitat Polit\`ecnica
de Catalunya}}

\centerline{\large \sc{ETSECCPB}}

%\centerline{\large \sc{Programa de Doctorat d'Enginyeria
%Civil}}

\vspace{0.3cm} \hrule \vspace{0.3cm} \centerline{\large
\sc{Treball Fi de M\`aster}}

\vspace{5cm}
\begin{center}
{\Large{\sc{\ttitulo}}}
\end{center}

\vspace{1.7cm}

\centerline{\Large by}

\vspace{0.8cm}

\centerline{\Large \sc{\Autor}}

\vfill

\hfill
\begin{tabular}{c}
{\Large \vspace{0.1cm} %Thesis Proposal
}\\
{\large Advisor: \Advisor} \\
\noalign{\vspace{0.1cm}} \hline \vspace{0.3cm} \null
{\large Barcelona, \Date}
\end{tabular}
\cleardoublepage \cleardoublepage

}

\newcommand{\doContents}
{
  \begin{singlespace}

  \phantomsection
  \addcontentsline{toc}{chapter}{\contentsname}
  \tableofcontents

  \cleardoublepage

  \phantomsection
  \addcontentsline{toc}{chapter}{\listfigurename}
  \listoffigures

  \cleardoublepage

  %\phantomsection
  %\addcontentsline{toc}{chapter}{\listtablename}
  %\listoftables

  %\clearpage\addcontentsline{toc}{chapter}{\boxesname}
  %\listof{Box}{\boxesname}

  \cleardoublepage

  \phantomsection
  \addcontentsline{toc}{chapter}{\listalgorithmname}
  \listofalgorithms

  \end{singlespace}
}



\newcommand{\doAbstractTitle}
{
  %\pagestyle{plain}

\thispagestyle{empty}

\vspace{-3cm} \centerline{\large \sc{Universitat Polit\`ecnica
de Catalunya}}

\centerline{\large \sc{Programa de Doctorat de Matem\`atica
Aplicada}}

%\centerline{\large \sc{Programa de Doctorat d'Enginyeria
%Civil}}

\vspace{0.3cm} \hrule \vspace{0.3cm} \centerline{\large
\sc{Departament de Matem\`atica Aplicada III}}

\vspace{5cm}
\begin{center}
{\Large{\sc{\ttitulo}}}\\
\rule{0.5\textwidth}{1pt}\\
{\Large \sc \abstractname}
\end{center}

\vspace{1.7cm}

\centerline{\Large by}

\vspace{0.8cm}

\centerline{\Large \sc{\Autor}}

\vfill

\hfill
\begin{tabular}{c}
{\Large Thesis Proposal}\\
{\large Advisor: \Advisor} \\
\noalign{\vspace{0.1cm}} \hline \vspace{0.3cm} \null
{\large Barcelona, \Date}
\end{tabular}
\cleardoublepage \cleardoublepage


  \thispagestyle{empty}

  \clearpage\addcontentsline{toc}{chapter}{\abstractname}

  \centerline{\sc{Universitat Polit\`ecnica
  de Catalunya}}

  \centerline{\sc{Programa de Doctorat de Matem\`atica
  Aplicada}}

  \vspace{0.3cm} \hrule \vspace{0.3cm}
  \centerline{\sc{Departament de Matem\`atica Aplicada III}}

  \vspace{0.75cm}

  \begin{center}
  {\sc{\Large \ttitulo}}\\
  \rule{0.5\textwidth}{1pt}\\
  {\sc \large \abstractname}\\[\baselineskip]
  by\\
  {\large \Autor}
  \end{center}

  \vspace{0.75cm}

  \hfill
  \begin{tabular}{r}
  {Thesis Proposal}\\
  {Advisor: \Advisor} \\
  \noalign{\vspace{0.1cm}} \hline \vspace{0.3cm} \null
  {Barcelona, \Date}
  \end{tabular}

  \vspace{2cm}
}

% New environments

\newenvironment{Abstract}
  {
    \clearpage\addcontentsline{toc}{chapter}{\abstractname}

    \begin{center}
    {\large \textsc{\abstractname}}\\[\baselineskip]
    \begin{singlespace}
    \Titulo \\[\baselineskip]
    \end{singlespace}
    \Autor \\[\baselineskip]
    \end{center}
  }
  {

  }

\newenvironment{Acknowledgements}
  {
    \clearpage\addcontentsline{toc}{chapter}{\acknowledgementname}
    \begin{center}
    {\large \textsc{\acknowledgementname}}\\[\baselineskip]
    \end{center}
  }
  {

  }

% New listOf definitions

\newcounter{todoListCounter}

\newcommand{\theTodoListCounter}
{
  \arabic{todoListCounter}
}

\newenvironment{todoList}[1]
{
  \refstepcounter{todoListCounter}
  \todo{#1}
  \begin{enumerate}\addtolength{\itemsep}{-0.5\baselineskip}
}
{
  \end{enumerate}
  \rule[1cm]{\textwidth}{1pt}
  \vspace{-1\baselineskip}
}

\newcommand{\listTodoName}
{
  There are \textcolor{red}{\theTodoListCounter} todo's
}

\newlistof{todo}{tmp}{\listTodoName}

\newcommand{\todo}[1]
{
	\refstepcounter{todo}
  \rule{\textwidth}{1pt}
	$\rhd$ \textbf{\large \theTodoListCounter} \textcolor{red}{ \texttt{TODO}}: {\emph{#1}}\\
  \rule[0.2cm]{\textwidth}{1pt}
  \vspace{-2\baselineskip} \newline
	\addcontentsline{tmp}{todo}{TODO \numberline{\thetodo:}#1}
}

% General definitions

\newcommand{\acknowledgementname}
{
  Acknowledgments
}

\newcommand{\Appendix}
{
  \fancyhead[LO]{\rightmark}
  \fancyhead[RE]{\sc{\leftmark}}

  \appendix
}

\newcommand{\boxesname}
{
  List of Boxes
}

\providecommand{\abstractname}
{
  Abstract
}

\newcommand{\done}
{
  {\Huge \textcolor{green}{\checkmark}}
}

% shortcuts

\newcommand{\x}{~$\times$~}
\newcommand{\fig}{Figure~}
\newcommand{\eref}[1]{Equation~(\ref{#1})}
\newcommand{\erefs}[1]{Equations~(\ref{#1})}
\newcommand{\fref}[1]{Figure~\ref{#1}}
\newcommand{\frefs}[1]{Figures~\ref{#1}}
\newcommand{\tref}[1]{Table~\ref{#1}}
\newcommand{\trefs}[1]{Tables~\ref{#1}}
\newcommand{\sref}[1]{Section~\ref{#1}}
\newcommand{\srefs}[1]{Sections~\ref{#1}}
\newcommand{\cref}[1]{Chapter~\ref{#1}}
\newcommand{\crefs}[1]{Chapters~\ref{#1}}
\newcommand{\aref}[1]{Appendix~\ref{#1}}
\newcommand{\tsty}{\textstyle}
\newcommand{\dsty}{\displaystyle}
\newcommand{\arrow}{~$\rightarrow$~}
\newcommand{\otheta}{\overline \theta}
\newcommand{\mathG}{\mathcal{G}}
\newcommand{\ds}{\displaystyle}
\newcommand{\ol}{\overline}

\newcommand{\Ker}{\mathrm{Ker \ }}
\renewcommand{\Im}{\mathrm{Im \ }}


% abbreviations

\newcommand{\eg}{e.g.,\xspace}
\newcommand{\ie}{i.e.,\xspace}
\newcommand{\etc}{etc.\@\xspace}
\newcommand{\cmatrixb}{\left\{ \begin{matrix}}
\newcommand{\cmatrixe}{\end{matrix} \right\}}

% general vector/matrix commands:

\newcommand{\tvm}[1]{\textbf{#1}}
\newcommand{\tvms}[1]{$\boldsymbol{#1}$\ }
\newcommand{\vm}[1]{\mathbf{#1}}
\newcommand{\vms}[1]{\boldsymbol{#1}}
\newcommand{\bsym}[1]{\boldsymbol{#1}}
\newcommand{\vect}[1]{\mathbf{#1}}
\newcommand{\mat}[1]{\mathbf{#1}}

% vector/matrix for space coordinates 'x' and 'y'

\newcommand{\vx}{\mathbf{x}}
\newcommand{\bx}{\boldsymbol{x}}
\newcommand{\vxI}{\mathbf{x}_{I}}
\newcommand{\xI}{x_{I}}
\newcommand{\yI}{y_{I}}
\newcommand{\hvx}{\hat{\mathbf{x}}}
\newcommand{\hx}{\hat{x}}
\newcommand{\hy}{\hat{y}}

% Shortcuts for making slides

\newcommand{\fontone}{\bfseries\Large}
\newcommand{\fonttwo}{\bfseries\large}
%\newcommand{\fontone}{\scshape\Large}
%\newcommand{\fonttwo}{\scshape\large}
\newcommand{\fontthree}{\bfseries}
\newcommand{\bc}{\begin{center}}
\newcommand{\ec}{\end{center}}
\newcommand{\bitem}{\begin{itemize}}
\newcommand{\eitem}{\end{itemize}}


%--My own definitions--

\newcommand{\reals}{{\mathbb R}}
\newcommand{\bfalfi}{\mbox{\boldmath$\alpha$\unboldmath$_i$}}
\newcommand{\bfalfj}{\mbox{\boldmath$\alpha$\unboldmath$_j$}}
\newcommand{\remark}[2]{\vspace{0.1cm}
\indent {\bf Remark #1:}  {#2}  \vspace{0.1cm}}
\newcommand{\invisible}[1]{}
\newcommand{\bgl}[1]{\mbox{\boldmath$#1$\unboldmath}}
\newcommand{\parti}[2]{\frac{\partial #1}{\partial #2}}

\newcommand{\listOfTodo}
{
  \ifthenelse{\theTodoListCounter > 0}
  {
    \phantomsection
    \addcontentsline{toc}{chapter}{Todo's}
    \cleardoublepage
    \listoftodo
  }
  {

  }
}

\theoremstyle{plain}
%{\theorembodyfont{\upshape}

\newtheorem{Rem}{Remark}[chapter]
\newtheorem{Proposition}{Proposition}[chapter]
\theoremstyle{definition}
\newtheorem{Proof}{Proof}[chapter]

\floatstyle{boxed}
\newfloat{Box}{tbp}{bux}[chapter]

%--My own definitions--

\newcommand{\C}
{
  @{}c@{}
}

\renewcommand{\algorithmicensure}{\textbf{Return :}}

\newcommand{\IfNot}[1]
{
  \If{\textbf{not} #1}
}

\newcommand{\Not}
{\textbf{not \ }}

\newcommand{\textproc}[1]
{\texttt{#1}}

\newcommand{\dotProduct}[2]
{
  <#1,#2>
}

\newcommand{\comp}[1]
{
  \overline{#1}
}

\newcommand{\define}[1]{\emph{#1}}

% color definitions

\definecolor{Olive}{rgb}{.8,0.8,0.0}
\definecolor{Blue}{rgb}{.6,0.8,1}
\definecolor{Orange}{rgb}{1,0.8,0}
\definecolor{Red}{rgb}{1,0.5,0.5}
\definecolor{Purple}{rgb}{.6,0.6,1}
\definecolor{Green}{rgb}{.2,0.8,0.4}

%item definition

\newcommand{\Item}[1]
{\item \emph{#1}}

% colored items

\newcommand{\coloredItem}[2]
{
  \item[\textcolor{#1}{$\bullet$}] \textbf{#2}
}

% new \includegraphics that looks for
% figures in a selected path

\newcommand{\includeGraphics}[2][]
{
  \includegraphics[#1]{\figuresPath/#2}
}


% my commands - Cesar A. Rivas
\newcommand{\reyn}{\mathrm{Re}}
\newcommand{\myttt}[1]{\texttt{#1}}

\newcommand{\myfirst}{1${}^{\mathrm{st}}$ }
\newcommand{\mysecond}{2${}^{\mathrm{nd}}$ }
\newcommand{\mythrid}{3${}^{\mathrm{rd}}$ }

%\newcommand{\mycomment}[1]{\texttt{\color{red}{#1}}}


\newcommand{\mypartial}[2]{\frac{\partial #1}{\partial #2}}
\newcommand{\mypartialtwo}[2]{\frac{\partial^{2}#1}{\partial #2^{2}}}
\newcommand{\mypartialcross}[3]{\frac{\partial^{2}#1}{{\partial #2 \partial #3}}}
\newcommand{\myfull}[2]{{\mathrm{d} #1 \over \mathrm{d} #2}}
\newcommand{\myfulltwo}[2]{{\mathrm{d}^{2} #1 \over \mathrm{d} #2^{2}}}
\newcommand{\mymaterial}[2]{{\mathrm{D} #1 \over \mathrm{D} #2}}

\newcommand{\mycomment}[1]{\texttt{\color{red}{#1}}}
\newcommand{\mymat}[1]{\boldsymbol{#1}}

\newcommand{\mySubfigureTwo}[2]{
    \begin{figure}[t!]
        \centering
        \subfigure[]{\label{fig:#1-1}\includeGraphics[trim =43mm 95mm 42mm 95mm, clip, width=0.475\textwidth]{figures/#1-1.pdf}}
        \subfigure[]{\label{fig:#1-2}\includeGraphics[trim =43mm 95mm 42mm 95mm, clip, width=0.475\textwidth]{figures/#1-2.pdf}}
        \caption[#2]{#2. \subref{fig:#1-1} Giuliani's Method. \subref{fig:#1-2} Shape-based Smoother.}
        \label{fig:#1}
    \end{figure}	
}

\newcommand{\mySubfigureFour}[2]{
    \begin{figure}[t!]
        \centering
        \subfigure[]{\label{fig:#1-before}\includeGraphics[trim = 52mm 87mm 52mm 87mm, clip, width=0.4\textwidth]{figures/#1-before.pdf}}
        \subfigure[]{\label{fig:#1-after}\includeGraphics[trim = 52mm 87mm 52mm 87mm, clip, width=0.4\textwidth]{figures/#1-after.pdf}}\\
        \subfigure[]{\label{fig:#1-before-hist}\includeGraphics[trim = 38mm 84mm 40mm 88mm, clip, width=0.4\textwidth]{figures/#1-before-hist.pdf}}
        \subfigure[]{\label{fig:#1-after-hist}\includeGraphics[trim = 38mm 84mm 40mm 88mm, clip, width=0.4\textwidth]{figures/#1-after-hist}}
        \caption[#2]{#2. Mesh configuration before \subref{fig:#1-before} and after \subref{fig:#1-after} smoothing. Shape quality histogram before \subref{fig:#1-before-hist} and after \subref{fig:#1-after-hist} smoothing.}
        \label{fig:#1}
    \end{figure}	
}

\newcommand{\mySubfigureFourB}[2]{
    \begin{figure}[b!h]
        \centering
        \subfigure[]{\label{fig:#1-before}\includeGraphics[trim = 52mm 87mm 52mm 87mm, clip, width=0.4\textwidth]{figures/#1-before.pdf}}
        \subfigure[]{\label{fig:#1-after}\includeGraphics[trim = 52mm 87mm 52mm 87mm, clip, width=0.4\textwidth]{figures/#1-after.pdf}}\\
        \subfigure[]{\label{fig:#1-before-hist}\includeGraphics[trim = 38mm 84mm 40mm 88mm, clip, width=0.4\textwidth]{figures/#1-before-hist.pdf}}
        \subfigure[]{\label{fig:#1-after-hist}\includeGraphics[trim = 38mm 84mm 40mm 88mm, clip, width=0.4\textwidth]{figures/#1-after-hist}}
        \caption[#2]{#2. Mesh configuration before \subref{fig:#1-before} and after \subref{fig:#1-after} smoothing. Shape quality histogram before \subref{fig:#1-before-hist} and after \subref{fig:#1-after-hist} smoothing.}
        \label{fig:#1}
    \end{figure}	
}

\newcommand{\mySubfigureFourAVS}[3]{
    \begin{figure}[t!]
        \centering
        \subfigure[]{\label{fig:#1-before}\includeGraphics[trim = #3, clip, width=0.4
    \textwidth]{figures/#1-before.pdf}}
        \subfigure[]{\label{fig:#1-after}\includeGraphics[trim = #3, clip, width=0.4
    \textwidth]{figures/#1-after.pdf}}\\
        \subfigure[]{\label{fig:#1-before-hist}\includeGraphics[trim = 40mm 80mm 40mm 88mm, clip, 
    width=0.4\textwidth]{figures/#1-before-hist.pdf}}
        \subfigure[]{\label{fig:#1-after-hist}\includeGraphics[trim = 40mm 80mm 40mm 88mm, clip, 
    width=0.4\textwidth]{figures/#1-after-hist}}
        \caption[#2]{#2. Mesh configuration before \subref{fig:#1-before} and after \subref{fig:#1-after} smoothing. Shape quality histogram before \subref{fig:#1-before-hist} and after \subref{fig:#1-after-hist} smoothing.}
        \label{fig:#1}
    \end{figure}	
}

\newcommand{\mySubfigureFourAVSB}[3]{
    \begin{figure}[b!]
        \centering
        \subfigure[]{\label{fig:#1-before}\includeGraphics[trim = #3, clip, width=0.4
    \textwidth]{figures/#1-before.pdf}}
        \subfigure[]{\label{fig:#1-after}\includeGraphics[trim = #3, clip, width=0.4
    \textwidth]{figures/#1-after.pdf}}\\
        \subfigure[]{\label{fig:#1-before-hist}\includeGraphics[trim = 40mm 80mm 40mm 88mm, clip, 
    width=0.4\textwidth]{figures/#1-before-hist.pdf}}
        \subfigure[]{\label{fig:#1-after-hist}\includeGraphics[trim = 40mm 80mm 40mm 88mm, clip, 
    width=0.4\textwidth]{figures/#1-after-hist}}
        \caption[#2]{#2. Mesh configuration before \subref{fig:#1-before} and after \subref{fig:#1-after} smoothing. Shape quality histogram before \subref{fig:#1-before-hist} and after \subref{fig:#1-after-hist} smoothing.}
        \label{fig:#1}
    \end{figure}	
}

\newcommand{\mySubfigureTwoAVS}[3]{
    \begin{figure}[t!]
        \centering
        \subfigure[]{\label{fig:#1-before-comp}\includeGraphics[trim = #3, clip, width=0.4
    \textwidth]{figures/#1-before.pdf}}
        \subfigure[]{\label{fig:#1-after-comp}\includeGraphics[trim = #3, clip, width=0.4
    \textwidth]{figures/#1-after.pdf}}
        \caption[#2]{#2. Mesh configuration before \subref{fig:#1-before} and after \subref{fig:#1-after} smoothing.}
        \label{fig:#1-comp}
    \end{figure}	
}

\newcommand{\myQualityTable}[9]{
    \begin{table}[h!]
    \centering
        \begin{tabular}{lcc}
        \hline
        & Before & After \\
        \hline
        Minimum   & #2 & #6 \\
        Maximum   & #3 & #7 \\
        Mean      & #4 & #8 \\
        Std. dev. & #5 & #9 \\
        \hline
        \end{tabular} 
    \caption{Shape-based quality statistical values for \fref{fig:#1}}
    \label{tbl:#1}
    \end{table}
}

\newcommand{\mySubfigureThree}[2]{
    \begin{figure}[t!]
        \centering
        \subfigure[]{\label{fig:#1-before}\includeGraphics[trim = 52mm 87mm 52mm 87mm, clip, width=0.3\textwidth]{figures/#1-before.pdf}}
        \subfigure[]{\label{fig:#1-after}\includeGraphics[trim = 52mm 87mm 52mm 87mm, clip, width=0.3\textwidth]{figures/#1-after.pdf}}
        \subfigure[]{\label{fig:#1-histo}\includeGraphics[trim = 38mm 84mm 40mm 88mm, clip, width=0.3\textwidth]{figures/#1-before-hist.pdf}}
        \caption[#2]{#2. Mesh configuration before \subref{fig:#1-before} and after \subref{fig:#1-after} smoothing. Quality histogram before and after \subref{fig:#1-histo}.}
        \label{fig:#1}
    \end{figure}	
}

